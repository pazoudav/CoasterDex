\documentclass{article}

\usepackage{amsmath, amsthm, amssymb, amsfonts}
\usepackage{thmtools}
\usepackage{graphicx}
\usepackage{setspace}
\usepackage{geometry}
\usepackage{float}
\usepackage{hyperref}
\usepackage[utf8]{inputenc}
\usepackage[english]{babel}
\usepackage{framed}
\usepackage[dvipsnames]{xcolor}
\usepackage{tcolorbox}

\colorlet{LightGray}{White!90!Periwinkle}
\colorlet{LightOrange}{Orange!15}
\colorlet{LightGreen}{Green!15}

\newcommand{\HRule}[1]{\rule{\linewidth}{#1}}

\declaretheoremstyle[name=Theorem,]{thmsty}
\declaretheorem[style=thmsty,numberwithin=section]{theorem}
\tcolorboxenvironment{theorem}{colback=LightGray}

\declaretheoremstyle[name=Proposition,]{prosty}
\declaretheorem[style=prosty,numberlike=theorem]{proposition}
\tcolorboxenvironment{proposition}{colback=LightOrange}

\declaretheoremstyle[name=Principle,]{prcpsty}
\declaretheorem[style=prcpsty,numberlike=theorem]{principle}
\tcolorboxenvironment{principle}{colback=LightGreen}

\setstretch{1.2}
\geometry{
    textheight=9in,
    textwidth=5.5in,
    top=1in,
    headheight=12pt,
    headsep=25pt,
    footskip=30pt
}

% ------------------------------------------------------------------------------

\begin{document}

% ------------------------------------------------------------------------------
% Cover Page and ToC
% ------------------------------------------------------------------------------

\title{ \normalsize \textsc{}
		\\ [2.0cm]
		\HRule{1.5pt} \\
		\huge \textbf{CoasterDex
		\HRule{2.0pt} \\ [0.6cm] \Large{software for detection and recognition of beer-coasters from images} \vspace*{10\baselineskip}}
		}
\date{}
\author{\textbf{Authors} \\ 
		David Pažout \& Dagur Elinór Kristinnson\\
		RU \\
		19.12.2023}

\maketitle
\newpage

\tableofcontents

\hspace{20pt}

\begin{abstract}
CoasterDex is a software for detection and classification of beercoasters in images and video.
It combines fine tuned Yolov5 based CNN with 
custom SIFT based instance level recognition algorithm 
to achieve high accuracy detection of beercoasters across varied environments.
\end{abstract}

\newpage

\section{Introduction}
% - problem statement & project idea. 
% - Relevant prior work reported by others.  Existing methods, software, and data sets that could be applied to your solution. 
% - How does your application, application area, or approach differ from the prior work?
\subsection{Problem Definition}
This problem is inspired by my friend who collects beercoasters. 
They have a collection of more then 200 unique beercoasters. 
Every time I'm in a pub and see a beercoaster I don't know if they already have it or not.
They aren't always available to be asked or they sometimes don't remember and bringing them a duplicate coaster is suboptimal.

Formal definition is as follows, we have a collection $C=\{c_1,...c_n | c_i \neq c_j;\ \forall i,j;\ i \neq j\}$ of images of unique beercoasters or coaster-scans from frontal view.
Given an input image $img$, the goal is to find all beercoasters $B=\{b_1...b_m\}$ present in the image 
and determine if a beercoaster $b \in B$ is also present in $C$. 
If $b$ isn't present in the collection we should add it to the collection ($C \leftarrow C \cup b$).

\subsection{Relevant Work}
We found a few papers tackling a similar or adjacent problems.
The oldest is a paper from 2011 solving postage stamp recognition from photo-scans \cite{}.
Since the paper was written before wide adoption of CNN it mainly uses custom, hand-crafted features and template matching for classification of relatively small number of stamps.
The authors in HotSpotter \cite{} try to match individual animals (not species) with past photos in a larger database with multiple instances of each individual.
In \cite{} CNN is used to tackle grocery product recognition. The authors approach the problem as classification of images into 100 plus classes.

Most of the approaches for the instance recognition were taken from two throughout overviews on the subject \cite{} \cite{},
where authors compare older SIFT based and newer CNN based approaches to this problem.

Working with small datasets can be tricky since it can lead to overfitted models. Using smaller models increases performance in these cases.\cite{} Data augmentation is also a powerful tool for these situations due to its ability to 
increase dataset size artificially. \cite{}

% add Finder background materials 


\section{Materials and Methods}
% - Methods, software, and datasets that you tested and selected. 
% - How did you adapt the methods for use in your application?
We decomposed the problem to two main parts: Finder and Matcher.
Finder is an image classification and localization algorithm and its goal is to find all
occurrences of beercoasters $B=\{b_1...b_m\}$ in the image $img$ with their appropriate bounding boxes.
Matcher is an instance level recognition/retrieval algorithm and its goal is to match 
a beercoaster from the collection $c \in C$ with a beercoaster $b$ present in the image $img$
or to determine that no matching beercoaster is present in the image.

\subsection{Finder - Coaster Detection}

To implement a coaster finding program the utilized method was yolov5 object detection model small version. The model was fine tuned
with a custom dataset specifically designed to train coaster detection. Roboflow was used to label the dataset and perform augmentations.
With a functional coaster detection model, opencv was used to extract the given bounding boxes and then returned to the matcher for 
integration.

\subsection{Matcher - Instance Recognition}


\subsection{Finder\&Matcher Integration}

We first input the image into the Finder, as output we receive a list of bounding boxes.
If the list is empty we input an unaltered image to the matcher.
Otherwise we do for each bounding box separably the following: 
we extract part of the image inside the bounding box, 
stretch the cropped image to a square - this helps with feature extraction since most coasters are of square-ish shape - 
and input the stretched image into the Matcher.
As the output we have a list of best matched coasters for each bounding box or the whole image. 



\section{Results}
% - Interpretation of the results. 
% - How do your results compare to the prior work?
% - How do they meet the needs of your application?
\subsection{Dataset}
% dataset creation and other stuff
In making a coaster finder dataset we lacked coasters due to coaster scarcity in Iceland where this project was worked on.
To solve this issue the images of coasters were printed on paper to create the percieved appearance of a coaster. The images of coasters
were found on a coaster collection database found on the website coasters.eu \cite{}. The printed images were cut out and roughly sized proportionally to real coasters. 
These cutouts were placed in various locations within the authors reach and photographed. The total number of coaster cutouts for the 
training dataset was 120 with some additional photos taken per coaster and 5 empty photos. This provides a dataset of 184 but 2 photos
including faces and 4 photos including beer in a glass were included later upon retraining to increase environment specific learning.
Heavy augmentation was also used to bring the total image count to 498 images. List of used augments is Flip, 90°Rotate, Crop, Shear, Grayscale, Hue, Brightness, Exposure, Blur,
Bounding Box: Rotation, Bounding Box: Shear, Bounding Box: Brightness, Bounding Box: Noise. The augments were applied randomly and automatically using roboflow.

\subsection{Finder}

To experiment with the finder results a new test dataset was created. The new dataset had 44 images with 50 coasters and various other objects located
within certain images. Two models were tested, both were fine tuned on yolov5s. One had 184 images, specifically missing the two images containing faces and four images containing beer.
The second model did contain the previously missing images and was heavily augmented. The results are as follows: 
Synthetic version:
    Recall = 0.96, Precision = 0.83
    Average confidence of correct detections = 0.74
Augmented version:
    Recall = 0.9, Precision = 0.85
    Average confidence of correct detections = 0.78

\subsection{Matcher}

\subsection{Finder\&Matcher}


\section{Discussion}


\section{Conclusion}
% - Conclusions and future work
% - What does your work tell you about the technical feasibility of achieving your goals?
% - Ideas for future work on the project or spin-offs.




% \begin{figure}[htbp]
%     \center
%     \includegraphics[scale=0.06]{img/photo.jpg}
%     \caption{Sydney, NSW}
% \end{figure}

\newpage

% ------------------------------------------------------------------------------
% Reference and Cited Works
% ------------------------------------------------------------------------------

\bibliographystyle{IEEEtran}
\bibliography{References.bib}

% ------------------------------------------------------------------------------

\end{document}